% !TeX spellcheck = pt_BR
%%%%%%%%%%%%%%%%%%%%%%%%%%%%%%%%%%%%%%%%%
% Transparência para aparesentações e aulas da Universidade Federal de Ouro Preto (UFOP) do Departamento de Sistemas de Informação e Computação (DECSI).
%
% Observação:
%	É necessário que seja criada uma pasta Imagens e a mesma deve possuir uma imagem chamda Logotipo de qualquer extensão (preferencialmnete uma que permita transparência e.g. PNG)
% 	para que seja possíel incluir o logotipo da UFOP no canto inferior direito de todas as transparências.
%
%
% LaTeX Template
% Versão 1.0 (27/10/15)
%
% Autor Original:
% João Pedro Santos de Moura 
% (https://github.com/jpmoura)
%
% Licença:
% CC BY-NC-SA 3.0 (http://creativecommons.org/licenses/by-nc-sa/3.0/)
%
%%%%%%%%%%%%%%%%%%%%%%%%%%%%%%%%%%%%%%%%%

%----------------------------------------------------------------------------------------
%	PACOTES E CONFIGURAÇÃO
%----------------------------------------------------------------------------------------

% Use o parâmetro 'handout' para gerar várias tansparências por páginas.
%\documentclass[handout,red]{beamer}
\documentclass[red]{beamer}

%\usepackage{pgfpages}			% Use este pacote caso utilize o parâmtro handout
\usepackage[portuguese]{babel}	% Pacote de vocabulário/internacionalização
\usepackage[utf8]{inputenc}		% Pacote para codificação em UTF-8 do texto de entrada
\usepackage{times} 				% Pacote para sar a fonte Times New Roman como fonte
\usepackage[T1]{fontenc}		% Pacote para codificação apropriada da fonte em artefatos DVI, PS e PDF
\usepackage{hyperref}			% Pacote para referência cruzada e produzir hiperlinks no documento gereado
\usepackage[alf]{abntex2cite}	% Citações padrão ABNT
\usepackage{tikz} 				% Necessário para uso de marca d'água caso requerido
\usepackage{listings}			% Pacote para inserção de códigos-fonte de arquivos externos
\usepackage{xcolor}         	% Pacote de cores

% Definição de estilo

\usetheme{Madrid} 
%\setbeamertemplate{background}{\tikz[overlay,remember picture]\node[opacity=0.4]at (current page.center){\includegraphics[width=2cm]{Imagens/Logomarca}};} % Marca d'água por trás do texto

% Definição de fonte
\usefonttheme[onlylarge]{structurebold}
\setbeamerfont*{frametitle}{size=\normalsize,series=\bfseries}

\setbeamertemplate{navigation symbols}{} % Símbolos de navegação próximo ao rodapé, TRUE entre colchetes significa ativado e vazio significa desativado

% Definição da logo, tanto caminho quanto tamanho
\pgfdeclareimage[height=1.4cm]{logo}{Imagens/Logotipo}
\logo{\pgfuseimage{logo}}


% Autor, título, e instituto

% Caso o título seja muito grande, passe como parâmetro entre colchetes uma versão reduzida do título
\title{Título}
%\title[titulo de rodapé]{Título}

% Caso o nome seja longo ou sejam vários autores, passe o(s) nome(s) reduzido(s) entre colchetes, como no título
\author{Autor}
%\author[Nomes no rodapé]{
%	Nomes dos autores\\
%	Vários nomes podem ser usados\\
%	Usando Quebra de linha}

\institute[] % Nomear o campo entre colcheter com a sigla da instituição. Ex.: UFOP. Ela aparecerá após o nome no 
{
	% Numerar entre chaves caso exista mais de uma instituição e use a numeração para referenciar a qual instituição pertence cada autor cajo sejam vários
	\inst{}
	Departamento de Computação e Sistemas de Informação\\
	Instituto de Ciências de Exatas e Aplicadas\\
	Universidade Federal de Ouro Preto
}

% Substitua o texto pela data desejada. Use o comando \today para usar a data da compilação.
\date{Data de apresentação}


%----------------------------------------------------------------------------------------
%	COMANDOS
%----------------------------------------------------------------------------------------
% Comando para cirar uma versão que possa ser impressa usando a opção handout como parâmtro do beamer. Nessa configuração ele irá gerar várias transparências por página para ser disponibilizada ao público-alvo.
% Lembre-se de usar o parãmtro 'handout' em documentclass, separado de 'red' por virgula e usar o pacote 'pdfpages'.

%\pgfpagesuselayout{2 on 1}[a4paper,border shrink=5mm] % Gera duas transparências por página 
%\pgfpagesuselayout{4 on 1}[a4paper,border shrink=5mm,landscape] % Gera quatro transparências por página

% Definição de cores em RGB
\definecolor{dkgreen}{rgb}{0,0.6,0}
\definecolor{gray}{rgb}{0.5,0.5,0.5}
\definecolor{mauve}{rgb}{0.58,0,0.82}
\definecolor{verde}{rgb}{0.25,0.5,0.35}
\definecolor{jpurple}{rgb}{0.5,0,0.35}
\definecolor{jstring}{RGB}{204,102,0}

% Parametrizando o estilo da linguagem Java usando pacote Listings
\lstset{
	language=Java,
	basicstyle=\ttfamily\small, 
	keywordstyle=\color{jpurple}\bfseries,
	stringstyle=\color{jstring},
	commentstyle=\color{verde},
	morecomment=[s][\color{blue}]{/**}{*/},
	extendedchars=true, 
	showspaces=false, 
	showstringspaces=false, 
	numbers=left,
	frame=tb,
	numberstyle=\tiny,
	breaklines=true, 
	backgroundcolor=\color{white}, 
	breakautoindent=true, 
	captionpos=b,
	xleftmargin=0pt,
	tabsize=4
}

%----------------------------------------------------------------------------------------
%	DOCUMENTO
%----------------------------------------------------------------------------------------


\begin{document}
	% Labels são necessários para usar hyperlinks junto com botões do pacote beamer.
	\label{Título}
	
	\begin{frame}
		
		\titlepage
	\end{frame}
	
	\begin{frame}[allowframebreaks]{Roteiro} % A opção 'allowframebrakes' possibilita que o índice seja divido em mais de uma transparência
		\tableofcontents
	\end{frame}
	
	\section{Introdução}
		\begin{frame}{Introdução}
			Este é um exemplo do artefato gerado a partir desse modelo. Neste modelo você encontra
			 também exemplos de uso de recursos do \textit{beamer}.
		\end{frame}
	
	\label{amb1}	
	\section{Ambientes}
		\begin{frame}{Exemplos de ambientes I}
			\begin{block}{Exemplo de bloco ordinário}
				Corpo de bloco normal.
			\end{block}
			
			\begin{alertblock}{Bloco de alerta}
				Corpo de um bloco de alerta.
			\end{alertblock}
			
			\begin{exampleblock}{Exemplo de ambiente \textit{exampleblock}}
				Corpo do ambiente ``examplebox".
			\end{exampleblock}
			
			\begin{example}
				A diferença entre um ambiente \textit{example} e um ambiente \textit{examplebox} é que
				no ambiente \textit{examplebox} o título é customizável enquanto no ambiente \textit{example} não. 
			\end{example}
			
		\end{frame}
		
		\label{amb2}
		\begin{frame}{Exemplos de ambientes II}
			Note que igualmente ao que acontece com o bloco \textit{example} os títulos dos blocos \textit{definition}, \textit{theorem} e \textit{corollary} não são traduzidos pelo pacote \textit{babel}.
			
			\begin{definition}
				O quadrado da hipotenusa é igual a soma do quadrado dos catetos. 
			\end{definition}
			
			\begin{theorem}[Pitágoras]
				$ a^2 + b^2 = c^2$
			\end{theorem}
			
			\begin{corollary}
				$ x + y = y + x  $
			\end{corollary}
			
			\begin{proof}
				$\omega +\phi = \epsilon $
			\end{proof}
		\end{frame}
		
		\begin{frame}{Exemplos de ambientes III}
			Ambiente \textit{description}:
			
			\begin{description}
				\item[API] Application Programming Interface
				\item[LAN] Local Area Network
				\item[ASCII] American Standard Code for Information Interchange
			\end{description}
		\end{frame}
		
		\section{Botões}
			\begin{frame}{Botões}
				O pacote \textit{beamer} permite o uso de botões junto com \textit{hyperlinks}.
				
				\begin{center}
					\hyperlink{amb1}{\beamergotobutton{Botão de \textit{vá para}}}\\
					\hyperlink{amb2}{\beamerskipbutton{Botão de avanço}}\\
					\hyperlink{title}{\beamerreturnbutton{Botão de retorno}}\\
					\hyperlink{perg}{\beamerbutton{Botão ordinário}}
				\end{center}				
			\end{frame}
		
		\section{Colunas}
			\begin{frame}{Usando colunas}
				Texto que existe antes do ambiente \textit{columns}
				
				\vspace*{5mm}
				
				\begin{columns}
					\column{0.5\textwidth}
					Texto na coluna 1\\
					Texto na coluna 1\\
					Texto na coluna 1\\
					Texto na coluna 1\\
					
					\column{0.5\textwidth}
					Texto na coluna 2\\
					Texto na coluna 2\\
					Texto na coluna 2\\
					Texto na coluna 2
				\end{columns}
				
				\vspace*{5mm}
				
				Fim das colunas. Pode-se inserir imagens, tabelas, listas, etc. dentro do ambiente \textit{columns}
			\end{frame}
		
		\section{Tabelas}
			\begin{frame}{Tabelas}
				Exemplo de tabela:
				\begin{table}
					\begin{tabular}{l | c | c | c | c }
						Competidor & Nado & Ciclismo & Corrida & Total \\
						\hline \hline
						João & 13:04 & 24:15 & 18:34 & 55:53 \\ 
						Naldo & 8:00 & 22:45 & 23:02 & 53:47\\
						Alex & 14:00 & 28:00 & n/a & n/a\\
						Sara & 9:22 & 21:10 & 24:03 & 54:35 
					\end{tabular}
					\caption{Resultados de Triatlon}
				\end{table}
			\end{frame}
			
		\section{Códigos}
			\subsection{Usando pacote listings}
				\begin{frame}{Código-fonte usando pacote \textit{listings}}
					Exemplo do uso do pacote \textit{listings} para inserir códigos-fonte externos.
					O código também pode ser escrito manualmente dentro do próprio arquivo Tex.
					Evite usar caracteres especiais e letras acentuadas, pois podem não ser exibidas.
				
					\lstinputlisting[language=java]{codigo.java}
				\end{frame}				
	
	% Transparências Finais
	\begin{frame}
		\label{perg}
		\begin{center} 
			\Huge Perguntas?
		\end{center} 
	\end{frame}
	
	\begin{frame}
		\begin{center} 
			\Huge Muito Obrigado!
		\end{center} 
	\end{frame}
	
	
	\begin{frame}{Referências}
		\nocite{Tantau2015,Cassidy2013}
		%\bibliographystyle{plain}   % Alterar o estilo da referência caso necessário
		\bibliography{referencias}   % Arquivo que contém as transparências
	\end{frame}
	
\end{document}